\section{Introduction}
Programming languages have to evolve to better address both techonology 
trends and developers needs. For instance, the Java programming language, which might be 
considered a reasonably recent language, presents features and constructs that 
differ significantly from its initial release in 1996. This kind of 
language evolution leads to a lot of dicussion regarding 
how this changes are embraced, used in practice, or ignored by the community of 
developers~\cite{}. One significant change to the Java language was 
the introduction of parametric polymorphism using Java Generics in 2004, which 
provides larger support for classes and methods generalization as well as 
improved support for software evolution. At that time, Java Generics 
was considered a significant improvement because it would simplify 
software construction by allowing the design of generic behavior using 
parametric types.

Several years later, Parnin et al. carried out an investigation to understand 
how Java Generics had been adopted in practice~\cite{}. 
In their study, they found out that over half of the projects and developers did not use generics at that time, and for those that did, the use was consistently narrow. They discovered, empirically, that generics were almost entirily used to either hold or traverse collections of objects in a type safe manner. 
%They concluded that an introduction of a language feature alone is not enough to assure adoption.
Nevertheless, the mentioned work did not answer a relevant question: is there 
any difference in the adoption of Java Generics when comparing two 
groups of systems based on the release date of this feature: one group of 
projects whose development started before Java SE 5.0 and one group of projects whose initial releases started after Java SE 5.0? 
{\color{red}Answering to this question might give \ldots}.

In this paper we first replicate the work of 
Pet et al.~\cite{} trying to answer aditional research questions~\ref{}. 

More recently, in 2014 a new version of the Java language was released (Java SE 8), introducing a long-waited feature that addresses some 
(limited) support for functional programming mechanisms: Lambda Expressions. In this paper we also characterize the adoption of that 
Java programming language construct, which might guide software 
developers to better understand the most common situations 
where Lambda Expressions should be used. In summary, the contributions of this paper are two-fold

\begin{itemize}
\item We replicate an existing study that investigates the 
adoption of Java Generics in open-source systems~\cite{}. Differently from 
the original work~\cite{}, our research also aims at 
understanding whether developers of recently developed 
systems embrace the use of Generics more extensively 
than developers of fully developed systems---whose initial 
releases started several years before Sun Microsystems 
launched Java SE 5.0 in 2004. 


\item We characterize how Java developers are using Lambda Expressions, 
a new Java language construct introduced in 2014 (Java SE 8). To 
the best of our knowledge, there is no other empirical study that investigates this issue.
\end{itemize}

\emph{Roadmap.} The remaining of this paper is organized 
as follows. {\color{red}In the next section \ldots} 
