\section{Related work}

After the release of a new language feature, it has become 
common to investigate its addoption as well as corresponding  
beneffits and drawbacks. This kind of investigation often uses 
some approach related to mining open-source software repositories. 
This section gives an overview of existing works that investigate 
the adoption of Java Generics and Java Lambda Expression.

\subsection{Research on the Adoption of Java Generics}

We found two main types of research works about Java Generics. The first one explores 
some claims about the use of Java Generics in differents contexts~\cite{}. The second type 
performs empirical studies that try to verify whether the initial claims and hypothesis 
about Java Generics hold or not. Actually, our goal here is to replicate one of these studies,
which empirically investigate the use of Java Generics~\cite{}. 

In that work, the authors conduct an empirical study w.r.t 
the use of Generics in open-source communities. Although other works 
analyse a similar question, we still consider that~\cite{} is the most detailed
research about the adoption of Java Generics, analysing the twenty \emph{most used} 
open-source projects at that time. Their study, published in 2009 (almost five years 
after the introduction of Java Generics), concludes that Java Generics was not 
being used in a relevant manner. Our goal is to replicate part of this study and evaluate how 
the adoption of Java Generics have changed so far. That is, we investigate whether 
the Java Generics usage might be still restricted to the reuse of the existing 
Java library of parameterized collections  or if developers are using it 
more broadly.

{\color{blue} mencionar Mining Billions of AST Nodes to Study Actual and
Potential Usage of Java Language Features }

\subsection{Lambda Expressions}

We found only a few works related to Java Lambda Expressions in our research, 
and they either focus on measuring the performance of this language feature~\cite{} or 
focus on the comparison of this language feature with Scala constructs~\cite{}, for instance. 
Nevertheless, none of these works investigates the real adoption of Lambda Expressions by 
practitioners. In our reserch, we found some work related to the use of lambda expressions in Java, 
\ldots {\color{red}seria bom discutir brevemente esses estudos.}
% focusing in comparisons with constructions in Scala 
% Language cite(work\_lambda1) , efficiency gains cite(work\_lambda2), 
% etc but none of this works investigating real adoption in projects.


